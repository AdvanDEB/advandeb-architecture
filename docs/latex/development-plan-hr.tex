\documentclass[12pt,a4paper]{article}
\usepackage[utf8]{inputenc}
\usepackage[croatian]{babel}
\usepackage{geometry}
\usepackage{hyperref}
\usepackage{enumitem}
\usepackage{graphicx}
\usepackage{fancyhdr}
\usepackage{tocloft}
\usepackage{xcolor}
\usepackage{amsmath}

\geometry{margin=2.5cm}

\definecolor{titlecolor}{RGB}{0,51,102}
\definecolor{sectioncolor}{RGB}{0,102,204}

\hypersetup{
    colorlinks=true,
    linkcolor=blue,
    filecolor=magenta,
    urlcolor=cyan,
    pdftitle={AdvanDEB Platforma - Plan Razvoja v3.0},
    pdfauthor={AdvanDEB Razvojni Tim},
}

\pagestyle{fancy}
\fancyhf{}
\fancyhead[L]{\small AdvanDEB Platforma - Plan Razvoja}
\fancyhead[R]{\small v3.0}
\fancyfoot[C]{\thepage}

\title{
    \textbf{\Huge \textcolor{titlecolor}{AdvanDEB Platforma}}\\
    \vspace{0.5cm}
    \Large Plan Razvoja\\
    \large Faza 0: Upravljanje Korisnicima i Autentifikacija\\
    \vspace{0.3cm}
    \normalsize Verzija 3.0 - Arhitektura Temeljena na Sposobnostima
}

\author{AdvanDEB Razvojni Tim}
\date{12. prosinca 2025.}

\begin{document}

\maketitle
\thispagestyle{empty}

\vspace{2cm}

\begin{abstract}
Ovaj dokument predstavlja sveobuhvatni plan razvoja AdvanDEB platforme, integriranog sustava za upravljanje biološkim znanjem i modeliranje temeljeno na jedinkama (IBM). Platforma se sastoji od dvije glavne komponente: Knowledge Builder za unos i upravljanje znanjem te Modeling Assistant za potporu IBM modeliranju. Ovaj izvještaj fokusira se na implementaciju Faze 0: uspostavljanje platformske autentifikacije, upravljanja korisnicima i infrastrukture za autorizaciju koristeći pojednostavljeni model uloga temeljen na sposobnostima.
\end{abstract}

\newpage
\tableofcontents
\newpage

\section{Izvršni Sažetak}

\subsection{Pregled Projekta}
AdvanDEB platforma je sveobuhvatni sustav dizajniran za potporu biološkim istraživanjima kroz upravljanje znanjem i modeliranje temeljeno na jedinkama. Platforma integrira dvije ključne komponente:

\begin{itemize}[leftmargin=*]
    \item \textbf{Knowledge Builder (KB)}: FastAPI + Vue.js sustav za unos, obradu i upravljanje biološkim znanjem iz literature, web izvora i strukturiranih podataka
    \item \textbf{Modeling Assistant (MA)}: Specijalizirana komponenta za pretraživanje znanja i zaključivanje u svrhu potpore radnim procesima modeliranja temeljenog na jedinkama
\end{itemize}

\subsection{Trenutna Faza Razvoja}
Projekt je trenutno u \textbf{Fazi 0: Upravljanje Korisnicima i Autentifikacija}. Ova temeljna faza uspostavlja sigurnost, kontrolu pristupa i infrastrukturu za suradnju potrebnu za sve buduće mogućnosti platforme.

\subsection{Ključna Arhitektonska Odluka: Pojednostavljenje v3.0}
Nakon početnih iteracija dizajna (v1.0 i v2.0), arhitektura je pojednostavljena na model temeljen na sposobnostima:

\begin{itemize}[leftmargin=*]
    \item \textbf{Od}: 6 različitih uloga (Administrator, Kustos Znanja, Recenzent Znanja, Operator Agenta, Analitičar Podataka, Istraživač Znanja)
    \item \textbf{Do}: 3 osnovne uloge (Administrator, Kustos Znanja, Istraživač Znanja) + 3 opcionalne sposobnosti (Pristup Agentima, Analitički Pristup, Status Recenzenta)
\end{itemize}

Ovo pojednostavljenje eliminira redundanciju uloga uz održavanje svih funkcionalnih zahtjeva kroz fleksibilniji model dozvola.

\section{Arhitektura Sustava}

\subsection{Komponente Platforme}

\subsubsection{advandeb-shared-utils}
Python paket koji pruža dijeljene uslužne programe za autentifikaciju i autorizaciju, eliminirajući dupliciranje koda između backend komponenti. Sadrži:
\begin{itemize}
    \item Generiranje i validaciju JWT tokena
    \item Upravljanje API ključevima
    \item Logiku provjere dozvola
    \item Modele korisnika (Pydantic)
    \item Uslužne programe za revizijsko bilježenje
    \item Pomoćnike za Google OAuth integraciju
\end{itemize}

\subsubsection{Knowledge Builder}
\textbf{Primarni pružatelj autentifikacije} za platformu. Sadrži:
\begin{itemize}
    \item Google OAuth 2.0 krajnje točke
    \item Sučelje za upravljanje korisnicima
    \item Radne procese za odobravanje uloga i sposobnosti
    \item Unos i obradu znanja
    \item Okvir AI agenata za automatiziranu ekstrakciju
    \item Konstrukciju grafa znanja
\end{itemize}

\subsubsection{Modeling Assistant}
Dijeli infrastrukturu autentifikacije s Knowledge Builderom:
\begin{itemize}
    \item Isti JWT tokeni vrijede za obje komponente
    \item Ista baza korisnika (MongoDB)
    \item Pretraživanje znanja iz KB podataka
    \item Sučelje za izgradnju scenarija
    \item Podrška za sastavljanje modela i simulaciju
\end{itemize}

\subsection{Arhitektura Autentifikacije}

\subsubsection{Jedinstvena Prijava (SSO)}
Korisnici se autentificiraju jednom putem Google OAuth 2.0 i dobivaju JWT tokene važeće za cijelu platformu. Iste vjerodajnice omogućuju pristup i Knowledge Builderu i Modeling Assistantu bez zasebne autentifikacije.

\subsubsection{Metode Autentifikacije}
\begin{enumerate}
    \item \textbf{Google OAuth 2.0}: Primarna metoda za korisnike web sučelja
    \item \textbf{API Ključevi}: Za programski pristup, s opsegom temeljenim na sposobnostima
    \item \textbf{JWT Tokeni}: Kratkoživući pristupni tokeni (1 sat) + tokeni za osvježavanje (30 dana)
\end{enumerate}

\subsubsection{Platformska Baza Korisnika}
Jedna MongoDB baza pohranjuje sve podatke vezane uz korisnike:
\begin{itemize}
    \item \texttt{users} kolekcija - Korisnički profili s base\_role i capabilities
    \item \texttt{capability\_requests} kolekcija - Radni procesi za odobravanje osnovnih uloga i sposobnosti
    \item \texttt{api\_keys} kolekcija - API ključevi važeći za cijelu platformu
    \item \texttt{audit\_logs} kolekcija - Potpuni revizijski trag za sve komponente
\end{itemize}

\section{Model Uloga i Dozvola v3.0}

\subsection{Osnovne Uloge}

\subsubsection{Administrator}
\textbf{Svrha}: Sistemska ovlast i konfiguracija platforme

\textbf{Dozvole}:
\begin{itemize}
    \item Potpuni pristup sustavu svim komponentama
    \item Upravljanje korisnicima (odobravanje uloga i sposobnosti)
    \item Konfiguracija sustava
    \item Unos početnog znanja (Day Zero)
    \item Poništavanje bilo koje odluke recenzije
    \item Pristup svim revizijskim zapisima
\end{itemize}

\subsubsection{Kustos Znanja}
\textbf{Svrha}: Kreator sadržaja i stručnjak domene

\textbf{Osnovne Dozvole}:
\begin{itemize}
    \item Učitavanje dokumenata (pojedinačno i grupno)
    \item Kreiranje činjenica i stiliziranih činjenica
    \item Izgradnja grafova znanja
    \item Kreiranje scenarija i modela (u MA)
    \item Uređivanje vlastitih doprinosa
    \item Pregled objavljenog znanja
\end{itemize}

\textbf{Opcionalne Sposobnosti} (moraju se zatražiti i odobriti):
\begin{itemize}
    \item \textbf{Pristup Agentima}: Pokretanje AI agenata, korištenje prilagođenih alata, pregled zapisa agenata
    \item \textbf{Analitički Pristup}: Napredni upiti, masovni izvoz, generiranje API ključeva
    \item \textbf{Status Recenzenta}: Pristup redu za recenziju, odobravanje/odbijanje znanja, kontrola kvalitete
\end{itemize}

\subsubsection{Istraživač Znanja}
\textbf{Svrha}: Korisnik samo za čitanje za pregledavanje znanja

\textbf{Dozvole}:
\begin{itemize}
    \item Pregledavanje i pretraživanje objavljenog znanja
    \item Pregled grafova znanja
    \item Pregled objavljenih modela i scenarija (u MA)
    \item Kreiranje privatnih bilješki
    \item Izvoz ograničenih skupova podataka (za osobnu uporabu)
    \item Spremanje upita za pretraživanje
\end{itemize}

\subsection{Radni Proces Zahtjeva za Sposobnosti}

\subsubsection{Novi Korisnik - Zahtjev za Osnovnu Ulogu}
\begin{enumerate}
    \item Korisnik se prijavljuje putem Googlea
    \item Status: \texttt{pending\_approval}, base\_role: \texttt{null}
    \item Korisnik ispunjava obrazac za zahtjev uloge (odabir Kustos ili Istraživač)
    \item Navodi pripadnost, područje istraživanja, obrazloženje
    \item Administrator pregleda i odobrava/odbija
    \item Korisnik prima obavijest e-mailom
    \item Pristup dodijeljen na temelju odobrene osnovne uloge
\end{enumerate}

\subsubsection{Postojeći Kustos - Zahtjev za Sposobnost}
\begin{enumerate}
    \item Kustos se prijavljuje s osnovnim pristupom
    \item Stranica profila prikazuje "Zatraži dodatne sposobnosti"
    \item Korisnik odabire željene sposobnosti:
    \begin{itemize}
        \item Pristup Agentima
        \item Analitički Pristup
        \item Status Recenzenta
    \end{itemize}
    \item Navodi obrazloženje za svaku sposobnost
    \item Administrator pregleda
    \item Sposobnosti dodane korisničkom profilu
    \item Korisnik odmah dobiva nove dozvole
\end{enumerate}

\subsection{Razrješavanje Dozvola}

Dozvole se izračunavaju na temelju:
\begin{equation}
\text{Korisničke Dozvole} = \text{Dozvole Osnovne Uloge} \cup \text{Dozvole Sposobnosti}
\end{equation}

\textbf{Primjeri}:
\begin{itemize}
    \item Kustos (samo osnovna): Može kreirati/uređivati znanje
    \item Kustos + Pristup Agentima: Može kreirati znanje I pokretati agente
    \item Kustos + Analitički Pristup + Status Recenzenta: Može kreirati, izvoziti I recenzirati
    \item Administrator: Ima sve dozvole neovisno o sposobnostima
\end{itemize}

\section{Trenutna Faza: Plan Razvoja Faze 0}

\subsection{Pregled Faze}

\textbf{Cilj}: Uspostaviti potpunu infrastrukturu za autentifikaciju, autorizaciju i upravljanje korisnicima za cijelu AdvanDEB platformu.

\textbf{Ishod}: Potpuno autentificirana platforma s 3 osnovne uloge + 3 sposobnosti, jedinstvena prijava kroz KB i MA, Google OAuth integracija, API ključevi, radni proces recenzije znanja i mogućnost Day Zero unosa.

\subsection{Plan Implementacije}

\subsubsection{Faza 1: Temelj}
\textbf{Fokus}: Backend sustav autentifikacije

\textbf{Zadaci}:
\begin{enumerate}
    \item Kreirati \texttt{advandeb-shared-utils} repozitorij i strukturu paketa
    \item Implementirati generiranje i validaciju JWT tokena
    \item Implementirati Google OAuth 2.0 klijent
    \item Kreirati User, CapabilityRequest, APIKey, AuditLog modele (Pydantic)
    \item Implementirati funkcije provjere dozvola (has\_base\_role, has\_capability)
    \item Postaviti uslužne programe za MongoDB vezu
    \item Kreirati funkcije za revizijsko bilježenje
    \item Napisati jedinične testove za uslužne programe autentifikacije
    \item Postaviti CI/CD za dijeljeni paket
\end{enumerate}

\textbf{Ishod}: Funkcionalni \texttt{advandeb-shared-utils} paket spreman za integraciju

\subsubsection{Faza 2: Backend Upravljanja Korisnicima}
\textbf{Fokus}: Integracija Knowledge Builder backenda

\textbf{Zadaci}:
\begin{enumerate}
    \item Dodati \texttt{advandeb-shared-utils} zavisnost KB backendu
    \item Kreirati \texttt{/auth} usmjerivač (prijava, povratni poziv, odjava, osvježavanje)
    \item Kreirati \texttt{/users} usmjerivač (profil, ažuriranje)
    \item Kreirati \texttt{/capability-requests} usmjerivač (kreiranje, lista, odobravanje/odbijanje)
    \item Kreirati \texttt{/api-keys} usmjerivač (generiranje, lista, opoziv)
    \item Implementirati AuthMiddleware za sve postojeće rute
    \item Implementirati RateLimiter middleware
    \item Implementirati AuditLogger middleware
    \item Kreirati UserService, RoleService, APIKeyService
    \item Postaviti MongoDB kolekcije: users, capability\_requests, api\_keys, audit\_logs
    \item Konfigurirati Google OAuth vjerodajnice
    \item Implementirati sustav obavijesti e-mailom
\end{enumerate}

\textbf{Ishod}: Potpuno funkcionalni backend za autentifikaciju i upravljanje korisnicima

\subsubsection{Faza 3: Integracija Frontenda}
\textbf{Fokus}: Knowledge Builder frontend

\textbf{Zadaci}:
\begin{enumerate}
    \item Kreirati Login View s Google OAuth gumbom
    \item Kreirati Profile View (prikaz informacija korisnika, osnovne uloge, sposobnosti)
    \item Kreirati Role Request View (za nove korisnike)
    \item Kreirati Capability Request View (za postojeće kustose)
    \item Kreirati Administrator Dashboard (lista korisnika, zahtjevi na čekanju)
    \item Kreirati API Key Management View (generiranje, pregled, opoziv)
    \item Implementirati Auth Store (Pinia/Vuex) s upravljanjem tokenima
    \item Dodati Axios presretače za ubacivanje JWT tokena
    \item Dodati automatsku logiku osvježavanja tokena
    \item Ažurirati sve postojeće prikaze s renderiranjem temeljenim na dozvolama
    \item Dodati rukovanje greškama za 401/403 odgovore
    \item Implementirati upite "Zatraži pristup" za nedostatne dozvole
\end{enumerate}

\textbf{Ishod}: Potpuni autentificirani frontend s jedinstvenom prijavom

\subsubsection{Faza 4: Radni Proces Recenzije}
\textbf{Fokus}: Sustav validacije znanja

\textbf{Zadaci}:
\begin{enumerate}
    \item Dodati \texttt{status} polje svim entitetima znanja (facts, stylized\_facts, graphs, documents)
    \item Implementirati prijelaze statusa: draft → pending\_review → published/rejected/changes\_requested
    \item Kreirati \texttt{/reviews} usmjerivač (red, odobravanje, odbijanje, zahtjev-izmjena)
    \item Kreirati ReviewService s poslovnom logikom
    \item Kreirati Review Queue View (za korisnike sa Status Recenzenta)
    \item Dodati značke statusa prikazima liste znanja
    \item Implementirati logiku dodjele recenzenata
    \item Dodati praćenje povijesti recenzija
    \item Kreirati nadzornu ploču recenzenta sa statistikom
    \item Dodati obavijesti e-mailom za promjene statusa recenzije
    \item Spriječiti samo-recenziju (korisnici ne mogu recenzirati vlastite doprinose)
\end{enumerate}

\textbf{Ishod}: Funkcionalni sustav peer recenzije za kontrolu kvalitete znanja

\subsubsection{Faza 5: Day Zero i Migracija}
\textbf{Fokus}: Početni unos znanja i migracija podataka

\textbf{Zadaci}:
\begin{enumerate}
    \item Kreirati Day Zero radni proces grupnog unosa
    \item Dodati \texttt{is\_day\_zero} oznaku entitetima znanja
    \item Implementirati krajnje točke za kreiranje Day Zero (samo admin)
    \item Dodati značke "Temeljno Znanje" u korisničkom sučelju
    \item Migrirati postojećih 1.300 PDF-ova iz /papers direktorija
    \item Kreirati skriptu za migraciju za naslijeđene podatke
    \item Dodati metapodatke o pripisivanju migriranom sadržaju
    \item Auto-odobravanje Day Zero sadržaja (preskakanje recenzije)
    \item Kreirati nadzornu ploču za upravljanje Day Zero
    \item Dodati grupno označavanje za Day Zero sadržaj
    \item Testirati i validirati sve migrirane podatke
\end{enumerate}

\textbf{Ishod}: Platforma popunjena temeljnim znanjem, naslijeđeni podaci migrirani

\subsubsection{Faza 6: MA Integracija i Dotjerivanje}
\textbf{Fokus}: Autentifikacija Modeling Assistanta i završno testiranje

\textbf{Zadaci}:
\begin{enumerate}
    \item Dodati \texttt{advandeb-shared-utils} zavisnost MA backendu
    \item Implementirati middleware za autentifikaciju u MA koristeći dijeljenu biblioteku
    \item Dodati JWT validaciju tokena svim MA rutama
    \item Implementirati provjere dozvola za MA-specifične operacije (kreiranje scenarija, pokretanje simulacija)
    \item Ažurirati MA frontend za korištenje dijeljenog Auth Store
    \item Testirati međukomponentnu autentifikaciju (KB → MA s istim tokenom)
    \item Dodati polje komponente revizijskim zapisima ("knowledge\_builder" vs "modeling\_assistant")
    \item Kreirati sveobuhvatne integracijske testove
    \item Izvršiti sigurnosnu reviziju (istek tokena, granice dozvola, ograničavanje stope)
    \item Testiranje opterećenja (autentificirati 100+ istovremenih korisnika)
    \item Napisati korisničku dokumentaciju (vodič za autentifikaciju, vodič za zahtjev sposobnosti)
    \item Napisati administratorsku dokumentaciju (upravljanje korisnicima, radni procesi odobravanja)
    \item Napisati razvojnu dokumentaciju (dodavanje novih dozvola, proširivanje sposobnosti)
    \item Završne popravke grešaka i dotjerivanje
\end{enumerate}

\textbf{Ishod}: Jedinstvena platforma s potpunom autentifikacijom kroz KB i MA

\section{Tehnički Detalji Implementacije}

\subsection{Shema Baze Podataka}

\subsubsection{users Kolekcija}
\begin{verbatim}
{
  "_id": ObjectId,
  "google_id": string,          // Jedinstveni
  "email": string,
  "name": string,
  "picture_url": string,
  "base_role": string,          // "administrator", "knowledge_curator",
                                //  "knowledge_explorator"
  "capabilities": [string],     // ["agent_access", "analytics_access",
                                //  "reviewer_status"]
  "status": string,             // "active", "suspended", "pending_approval"
  "created_at": datetime,
  "updated_at": datetime,
  "last_login": datetime,
  "login_count": int,
  "metadata": {
    "affiliation": string,
    "research_area": string,
    "orcid": string
  }
}
\end{verbatim}

\subsubsection{capability\_requests Kolekcija}
\begin{verbatim}
{
  "_id": ObjectId,
  "user_id": ObjectId,
  "request_type": string,       // "base_role" ili "capability"
  
  // Za zahtjeve osnovne uloge
  "requested_base_role": string,
  "current_base_role": string,
  
  // Za zahtjeve sposobnosti
  "requested_capabilities": [string],
  "current_capabilities": [string],
  
  "justification": string,
  "form_data": dict,
  "status": string,             // "pending", "approved", "rejected"
  "created_at": datetime,
  "reviewed_by": ObjectId,
  "reviewed_at": datetime,
  "review_notes": string
}
\end{verbatim}

\subsubsection{api\_keys Kolekcija}
\begin{verbatim}
{
  "_id": ObjectId,
  "user_id": ObjectId,
  "key_hash": string,           // SHA-256 običnog ključa
  "key_prefix": string,         // "advk_abc12345"
  "name": string,
  "scopes": [string],           // Auto-dodijeljeno na temelju
                                // base_role + capabilities korisnika
  "status": string,             // "active", "revoked", "expired"
  "created_at": datetime,
  "expires_at": datetime,
  "last_used_at": datetime,
  "rate_limit": {
    "requests_per_minute": int,
    "requests_per_day": int
  }
}
\end{verbatim}

\subsubsection{audit\_logs Kolekcija}
\begin{verbatim}
{
  "_id": ObjectId,
  "user_id": ObjectId,
  "action": string,             // "create_fact", "approve_knowledge", itd.
  "resource_type": string,      // "fact", "document", "scenario", itd.
  "resource_id": ObjectId,
  "component": string,          // "knowledge_builder" ili "modeling_assistant"
  "details": dict,
  "ip_address": string,
  "user_agent": string,
  "auth_method": string,        // "jwt", "api_key"
  "timestamp": datetime
}
\end{verbatim}

\subsection{API Krajnje Točke}

\subsubsection{Rute Autentifikacije}
\begin{itemize}
    \item \texttt{GET /auth/login} - Preusmjeravanje na Google OAuth
    \item \texttt{GET /auth/callback} - Rukovatelj povratnog poziva OAuth
    \item \texttt{POST /auth/logout} - Poništavanje tokena
    \item \texttt{POST /auth/refresh} - Osvježavanje pristupnog tokena
\end{itemize}

\subsubsection{Rute Upravljanja Korisnicima}
\begin{itemize}
    \item \texttt{GET /users/me} - Dohvati trenutni korisnički profil
    \item \texttt{PATCH /users/me} - Ažuriraj profil
    \item \texttt{GET /users} - Lista korisnika (samo admin)
    \item \texttt{GET /users/:id} - Dohvati detalje korisnika (samo admin)
\end{itemize}

\subsubsection{Rute Zahtjeva za Sposobnosti}
\begin{itemize}
    \item \texttt{POST /capability-requests} - Kreiraj novi zahtjev
    \item \texttt{GET /capability-requests} - Lista vlastitih zahtjeva
    \item \texttt{GET /capability-requests/pending} - Lista na čekanju (samo admin)
    \item \texttt{POST /capability-requests/:id/approve} - Odobri (samo admin)
    \item \texttt{POST /capability-requests/:id/reject} - Odbij (samo admin)
\end{itemize}

\subsubsection{Rute API Ključeva}
\begin{itemize}
    \item \texttt{POST /api-keys} - Generiraj novi ključ
    \item \texttt{GET /api-keys} - Lista vlastitih ključeva
    \item \texttt{DELETE /api-keys/:id} - Opozovi ključ
\end{itemize}

\subsubsection{Rute Recenzije}
\begin{itemize}
    \item \texttt{GET /reviews/queue} - Stavke na čekanju za recenziju (zahtijeva Status Recenzenta)
    \item \texttt{POST /reviews/:id/approve} - Odobri znanje
    \item \texttt{POST /reviews/:id/reject} - Odbij znanje
    \item \texttt{POST /reviews/:id/request-changes} - Zatraži izmjene
\end{itemize}

\subsection{Sigurnosne Razmatranja}

\subsubsection{Sigurnost Tokena}
\begin{itemize}
    \item JWT tajni ključ: 256-bitna slučajna vrijednost pohranjena u okolini
    \item Pristupni tokeni: Kratkoživući (1 sat) za ograničavanje prozora izloženosti
    \item Tokeni za osvježavanje: Dugoživući (30 dana) ali sigurno pohranjeni
    \item Rotacija tokena pri osvježavanju za sprječavanje napada ponavljanja
    \item JTI (JWT ID) za mogućnost opoziva tokena
\end{itemize}

\subsubsection{Sigurnost API Ključeva}
\begin{itemize}
    \item SHA-256 hash prije pohrane (obični ključ se nikad ne pohranjuje)
    \item Identifikacija temeljena na prefiksu (\texttt{advk\_...}) za brzo pretraživanje
    \item Ograničavanje stope: 200-500 zahtjeva/minutu ovisno o sposobnostima
    \item Automatski istek (30-90 dana)
    \item Podrška za bijelu listu IP adresa (opcionalno)
\end{itemize}

\subsubsection{Ograničavanje Stope}
\begin{itemize}
    \item Ograničenja po korisniku temeljena na sposobnostima
    \item Redis-bazirani algoritam token bucket
    \item Ograničenja: Istraživač (100 zah/min), Kustos (200 zah/min), Kustos+Analitika (500 zah/min)
    \item Dnevna ograničenja za grupne operacije
\end{itemize}

\subsubsection{Revizijsko Bilježenje}
\begin{itemize}
    \item Sve operacije pisanja zabilježene
    \item Svi događaji autentifikacije zabilježeni
    \item Sve promjene dozvola zabilježene
    \item Nepromjenjivi zapisi (samo dodavanje)
    \item Zadržavanje: minimalno 2 godine
\end{itemize}

\section{Strategija Testiranja}

\subsection{Jedinični Testovi}
\begin{itemize}
    \item Sve funkcije u \texttt{advandeb-shared-utils}
    \item Generiranje i validacija JWT
    \item Logika provjere dozvola
    \item Hash i validacija API ključeva
    \item Ciljana pokrivenost: 90\%+
\end{itemize}

\subsection{Integracijski Testovi}
\begin{itemize}
    \item Potpuni tokovi autentifikacije (OAuth, JWT, API ključevi)
    \item Međukomponentna autentifikacija (KB → MA)
    \item Provođenje dozvola na svim zaštićenim rutama
    \item Radni proces zahtjeva za sposobnosti
    \item Radni proces recenzije
\end{itemize}

\subsection{End-to-End Testovi}
\begin{itemize}
    \item Registracija korisnika kroz odobrenje
    \item Kustos zahtijeva sposobnosti
    \item Kreiranje znanja i recenzija
    \item Međukomponentni pristup (kreiranje u KB, pregled u MA)
\end{itemize}

\subsection{Sigurnosno Testiranje}
\begin{itemize}
    \item Pokušaj pristupa bez autentifikacije
    \item Pokušaj eskalacije privilegija
    \item Validacija isteka tokena
    \item Provođenje ograničavanja stope
    \item Pokušaji SQL injekcije (iako se koristi MongoDB)
    \item Testiranje XSS ranjivosti
\end{itemize}

\section{Plan Postavljanja}

\subsection{Razvojno Okruženje}
\begin{itemize}
    \item Lokalna MongoDB instanca
    \item Lokalni Ollama za LLM funkcionalnost
    \item Google OAuth testne vjerodajnice
    \item Lažni servis za e-mail
\end{itemize}

\subsection{Staging Okruženje}
\begin{itemize}
    \item Hostirana MongoDB (Atlas)
    \item Postavljeni backend (Docker kontejneri)
    \item Postavljeni frontend (Nginx)
    \item Prave Google OAuth vjerodajnice (testna domena)
    \item Pravi servis za e-mail (SendGrid/Mailgun)
\end{itemize}

\subsection{Produkcijsko Okruženje}
\begin{itemize}
    \item Proizvodni MongoDB klaster (replica set)
    \item Redis klaster za ograničavanje stope
    \item Opterećenjem balansirani backend serveri
    \item CDN za frontend resurse
    \item SSL/TLS certifikati
    \item Sigurnosna kopija i oporavak od katastrofe
    \item Praćenje i upozoravanje
\end{itemize}

\section{Procjena Rizika}

\subsection{Tehnički Rizici}

\subsubsection{Rizik: Kompleksna Logika Dozvola}
\textbf{Vjerojatnost}: Srednja\\
\textbf{Utjecaj}: Visok\\
\textbf{Ublažavanje}:
\begin{itemize}
    \item Sveobuhvatni jedinični testovi za funkcije dozvola
    \item Dokumentacija matrice dozvola
    \item Fokus pregleda koda na autorizacijskom kodu
\end{itemize}

\subsubsection{Rizik: Problemi OAuth Integracije}
\textbf{Vjerojatnost}: Niska\\
\textbf{Utjecaj}: Visok\\
\textbf{Ublažavanje}:
\begin{itemize}
    \item Rano integracijske testiranje
    \item Alternativa email/lozinka autentifikacija ako je potrebno
    \item Dobro dokumentirana OAuth konfiguracija
\end{itemize}

\subsubsection{Rizik: Uska Grla Performansi}
\textbf{Vjerojatnost}: Srednja\\
\textbf{Utjecaj}: Srednji\\
\textbf{Ublažavanje}:
\begin{itemize}
    \item Testiranje opterećenja tijekom završnih faza
    \item Redis cache za često pristupane podatke
    \item Indeksiranje baze na pretragama korisnika
\end{itemize}

\subsection{Projektni Rizici}

\subsubsection{Rizik: Povećanje Opsega}
\textbf{Vjerojatnost}: Srednja\\
\textbf{Utjecaj}: Visok\\
\textbf{Ublažavanje}:
\begin{itemize}
    \item Strogo pridržavanje opsega Faze 0
    \item Zahtjevi za funkcionalnosti odgođeni za Fazu 1
    \item Tjedni pregledi napretka
\end{itemize}

\subsubsection{Rizik: Zakašnjenja Projekta}
\textbf{Vjerojatnost}: Srednja\\
\textbf{Utjecaj}: Srednji\\
\textbf{Ublažavanje}:
\begin{itemize}
    \item Redovito praćenje prekretnica
    \item Rano identificiranje prepreka
    \item Fleksibilna alokacija resursa
\end{itemize}

\section{Kriteriji Uspjeha}

Faza 0 će se smatrati završenom kada:

\begin{enumerate}
    \item Sve metode autentifikacije funkcionalne (Google OAuth, JWT, API ključevi)
    \item Sve 3 osnovne uloge + 3 sposobnosti implementirane i testirane
    \item Radni proces zahtjeva za sposobnosti funkcionalan za osnovne uloge i sposobnosti
    \item Radni proces recenzije znanja operativan
    \item Postojećih 1.300 PDF-ova migrirano s Day Zero pripisivanjem
    \item Međukomponentna autentifikacija verificirana (KB $\leftrightarrow$ MA)
    \item Potpuno revizijsko bilježenje operativno
    \item Sigurnosna revizija prošla
    \item Korisnička dokumentacija potpuna
    \item Administratorska dokumentacija potpuna
    \item Svi jedinični, integracijski i E2E testovi prolaze
    \item Sustav rukuje s 100+ istovremenih korisnika
\end{enumerate}

\section{Sljedeće Faze}

Nakon uspješnog završetka Faze 0:

\subsection{Faza 1: Stabilizacija Knowledge Buildera}
\begin{itemize}
    \item Ojačavanje CRUD operacija znanja s dozvolama
    \item Stabilizacija okvira agenata s pripisivanjem korisnika
    \item Dodavanje sveobuhvatne pokrivenosti testovima
    \item Implementacija CI/CD pipeline
    \item Optimizacija performansi
\end{itemize}

\subsection{Faza 2: Prototip Modeling Assistanta}
\begin{itemize}
    \item Finaliziranje MA ugovora o integraciji znanja
    \item Implementacija izgradnje scenarija s dozvolama
    \item Kreiranje sučelja za sastavljanje modela
    \item Validacija end-to-end toka: znanje → modeliranje
\end{itemize}

\subsection{Faza 3: Poboljšana Integracija i UX}
\begin{itemize}
    \item Napredno pretraživanje prilagođeno slučajevima modeliranja
    \item Vizualno istraživanje znanja u MA
    \item Značajke suradnje (dijeljeni scenariji)
    \item Praćenje doprinosa više korisnika
\end{itemize}

\subsection{Faza 4: Proširenja i Dodaci}
\begin{itemize}
    \item Mehanizam dodataka za prilagođene alate
    \item Podrška za dodatne paradigme modeliranja
    \item Napredne značajke suradnje (radni prostori, timovi)
    \item Sustav povjerenja i reputacije
\end{itemize}

\section{Zaključak}

Faza 0 predstavlja kritičan temelj za AdvanDEB platformu, uspostavljajući sigurnost, kontrolu pristupa i infrastrukturu za suradnju. Pojednostavljena v3.0 arhitektura temeljena na sposobnostima pruža fleksibilnost uz smanjenje kompleksnosti u usporedbi s prethodnim dizajnima.

Po završetku, platforma će imati robustan, produkcijski spreman sustav autentifikacije koji se može skalirati za potporu budućim fazama.

Model temeljen na sposobnostima omogućuje korisnicima da povećavaju svoj pristup kako se njihove potrebe razvijaju, podržavajući cilj platforme poticanja suradnje uz održavanje kontrole kvalitete kroz radni proces recenzije.

\vspace{1cm}

\noindent\textbf{Verzija Dokumenta}: 3.0\\
\textbf{Datum}: 12. prosinca 2025.\\
\textbf{Status}: Spremno za Implementaciju\\
\textbf{Kontakt}: AdvanDEB Razvojni Tim

\end{document}
