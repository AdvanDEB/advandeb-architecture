% AdvandEB Development Environment Documentation
% Printable setup and environment guide.

\documentclass[11pt]{article}

\usepackage[margin=1in]{geometry}
\usepackage{hyperref}
\usepackage{enumitem}

\title{AdvandEB Development Environment}
\author{AdvandEB Project}
\date{\today}

\begin{document}

\maketitle

\section{Overview}

This document provides a printable guide to the shared development environment used by the AdvandEB project. All services share a single Conda environment named \texttt{advandeb}.

The main repositories are:

\begin{itemize}[noitemsep]
  \item \texttt{advandeb-knowledge-builder} -- knowledge ingestion, processing, and exploration.
  \item \texttt{advandeb-modeling-assistant} -- modeling-focused retrieval and reasoning (planned).
  \item \texttt{advandeb-architecture} -- system-level documentation, diagrams, and environment definition.
\end{itemize}

\section{Conda Environment}

\subsection{Creating the Environment}

The canonical \texttt{environment.yml} file lives in the \texttt{advandeb-architecture} repository. To create the environment:

\begin{verbatim}
conda env create -f environment.yml
\end{verbatim}

If the file changes and you need to update an existing environment:

\begin{verbatim}
conda env update -f environment.yml --prune
\end{verbatim}

\subsection{Activating the Environment}

To activate the shared environment:

\begin{verbatim}
conda activate advandeb
\end{verbatim}

This environment is used for:

\begin{itemize}[noitemsep]
  \item Back-end services (FastAPI applications).
  \item Command-line tools (e.g., \texttt{uvicorn}, \texttt{plantuml} if installed via Conda).
  \item Any Python-based utilities for data processing and analysis.
\end{itemize}

\section{Core External Services}

\subsection{MongoDB}

AdvandEB services use MongoDB for persistent storage.

\begin{itemize}[noitemsep]
  \item Default URI: \texttt{mongodb://localhost:27017}
  \item Typical environment variable: \texttt{MONGODB\_URL}
\end{itemize}

Start MongoDB using your preferred method (system service, Docker, or local binary).

\subsection{Ollama}

Ollama provides local hosting for large language models used by the Knowledge Builder and, in the future, the Modeling Assistant.

\begin{itemize}[noitemsep]
  \item Default base URL: \texttt{http://localhost:11434}
  \item Typical environment variable: \texttt{OLLAMA\_BASE\_URL}
\end{itemize}

Install Ollama following its official instructions and ensure that it is running before starting any LLM-dependent services.

\section{Running the Knowledge Builder}

\subsection{Back-end}

From the \texttt{advandeb-knowledge-builder/backend} directory:

\begin{verbatim}
conda activate advandeb
cp .env.example .env   # edit as needed
uvicorn main:app --host 0.0.0.0 --port 8000 --reload
\end{verbatim}

Important environment variables in \texttt{backend/.env}:

\begin{itemize}[noitemsep]
  \item \texttt{MONGODB\_URL} -- MongoDB connection string.
  \item \texttt{DATABASE\_NAME} -- database name (e.g., \texttt{advandeb\_knowledge\_builder\_kb}).
  \item \texttt{OLLAMA\_BASE\_URL} -- Ollama endpoint.
\end{itemize}

\subsection{Front-end}

From the \texttt{advandeb-knowledge-builder/frontend} directory:

\begin{verbatim}
npm install
npm run dev
\end{verbatim}

The front-end development server typically runs at \texttt{http://localhost:3000}.

\section{Running the Modeling Assistant (Planned)}

The Modeling Assistant will also use the shared \texttt{advandeb} environment. A likely pattern for the future back-end service is:

\begin{verbatim}
conda activate advandeb
# from advandeb-modeling-assistant/backend (future)
uvicorn main:app --host 0.0.0.0 --port 8001 --reload
\end{verbatim}

It will communicate with:

\begin{itemize}[noitemsep]
  \item The Knowledge Builder's HTTP APIs for knowledge and agent operations.
  \item MongoDB for its own collections (e.g., scenarios, models, runs).
\end{itemize}

\section{Diagram Rendering}

Architecture diagrams are stored as PlantUML files in \texttt{advandeb-architecture/diagrams}.

To render all diagrams (assuming \texttt{plantuml} is installed and available):

\begin{verbatim}
conda activate advandeb
cd advandeb-architecture/diagrams
plantuml *.puml
\end{verbatim}

This generates image files (for example, PNGs) that can be included in LaTeX documents such as the main architecture specification.

\section{Troubleshooting}

\begin{itemize}[noitemsep]
  \item If Python imports such as \texttt{fastapi} or \texttt{uvicorn} cannot be resolved, ensure that the \texttt{advandeb} environment is active and that it was created from the canonical \texttt{environment.yml}.
  \item If diagrams fail to render with ``command not found'', install PlantUML via Conda or your preferred method and re-run the rendering commands.
  \item For MongoDB or Ollama connection issues, verify that the services are running and that the URLs configured in your \texttt{.env} files match the actual hosts and ports.
\end{itemize}

\end{document}
