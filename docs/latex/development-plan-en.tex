\documentclass[12pt,a4paper]{article}
\usepackage[utf8]{inputenc}
\usepackage[english]{babel}
\usepackage{geometry}
\usepackage{hyperref}
\usepackage{enumitem}
\usepackage{graphicx}
\usepackage{fancyhdr}
\usepackage{tocloft}
\usepackage{xcolor}
\usepackage{amsmath}

\geometry{margin=2.5cm}

\definecolor{titlecolor}{RGB}{0,51,102}
\definecolor{sectioncolor}{RGB}{0,102,204}

\hypersetup{
    colorlinks=true,
    linkcolor=blue,
    filecolor=magenta,
    urlcolor=cyan,
    pdftitle={AdvanDEB Platform - Development Plan v3.0},
    pdfauthor={AdvanDEB Development Team},
}

\pagestyle{fancy}
\fancyhf{}
\fancyhead[L]{\small AdvanDEB Platform - Development Plan}
\fancyhead[R]{\small v3.0}
\fancyfoot[C]{\thepage}

\title{
    \textbf{\Huge \textcolor{titlecolor}{AdvanDEB Platform}}\\
    \vspace{0.5cm}
    \Large Development Plan\\
    \large Phase 0: User Management \& Authentication\\
    \vspace{0.3cm}
    \normalsize Version 3.0 - Capability-Based Architecture
}

\author{AdvanDEB Development Team}
\date{December 12, 2025}

\begin{document}

\maketitle
\thispagestyle{empty}

\vspace{2cm}

\begin{abstract}
This document presents the comprehensive development plan for the AdvanDEB Platform, an integrated system for biological knowledge management and individual-based modeling (IBM). The platform consists of two main components: the Knowledge Builder for knowledge ingestion and management, and the Modeling Assistant for IBM-based modeling support. This report focuses on Phase 0 implementation: establishing platform-wide authentication, user management, and authorization infrastructure using a simplified capability-based role model.
\end{abstract}

\newpage
\tableofcontents
\newpage

\section{Executive Summary}

\subsection{Project Overview}
The AdvanDEB Platform is a comprehensive system designed to support biological research through knowledge management and individual-based modeling. The platform integrates two core components:

\begin{itemize}[leftmargin=*]
    \item \textbf{Knowledge Builder (KB)}: A FastAPI + Vue.js system for ingesting, processing, and managing biological knowledge from literature, web sources, and structured data
    \item \textbf{Modeling Assistant (MA)}: A specialized component for knowledge retrieval and reasoning to support individual-based modeling workflows
\end{itemize}

\subsection{Current Development Phase}
The project is currently in \textbf{Phase 0: User Management \& Authentication}. This foundational phase establishes the security, access control, and collaboration infrastructure required for all future platform capabilities.

\subsection{Key Architectural Decision: v3.0 Simplification}
Following initial design iterations (v1.0 and v2.0), the architecture has been simplified to a capability-based model:

\begin{itemize}[leftmargin=*]
    \item \textbf{From}: 6 distinct roles (Administrator, Knowledge Curator, Knowledge Reviewer, Agent Operator, Data Analyst, Knowledge Explorator)
    \item \textbf{To}: 3 base roles (Administrator, Knowledge Curator, Knowledge Explorator) + 3 optional capabilities (Agent Access, Analytics Access, Reviewer Status)
\end{itemize}

This simplification eliminates role redundancy while maintaining all functional requirements through a more flexible permission model.

\section{System Architecture}

\subsection{Platform Components}

\subsubsection{advandeb-shared-utils}
A Python package providing shared authentication and authorization utilities, eliminating code duplication between backend components. Contains:
\begin{itemize}
    \item JWT token generation and validation
    \item API key management
    \item Permission checking logic
    \item User models (Pydantic)
    \item Audit logging utilities
    \item Google OAuth integration helpers
\end{itemize}

\subsubsection{Knowledge Builder}
\textbf{Primary Authentication Provider} for the platform. Hosts:
\begin{itemize}
    \item Google OAuth 2.0 endpoints
    \item User management interface
    \item Role and capability approval workflows
    \item Knowledge ingestion and processing
    \item AI agent framework for automated extraction
    \item Knowledge graph construction
\end{itemize}

\subsubsection{Modeling Assistant}
Shares authentication infrastructure with Knowledge Builder:
\begin{itemize}
    \item Same JWT tokens valid across both components
    \item Same user database (MongoDB)
    \item Knowledge retrieval from KB data
    \item Scenario building interface
    \item Model assembly and simulation support
\end{itemize}

\subsection{Authentication Architecture}

\subsubsection{Single Sign-On (SSO)}
Users authenticate once via Google OAuth 2.0 and receive JWT tokens valid across the entire platform. The same credentials provide access to both Knowledge Builder and Modeling Assistant without separate authentication.

\subsubsection{Authentication Methods}
\begin{enumerate}
    \item \textbf{Google OAuth 2.0}: Primary method for web UI users
    \item \textbf{API Keys}: For programmatic access, with capability-based scopes
    \item \textbf{JWT Tokens}: Short-lived access tokens (1 hour) + refresh tokens (30 days)
\end{enumerate}

\subsubsection{Platform-Wide User Database}
Single MongoDB database stores all user-related data:
\begin{itemize}
    \item \texttt{users} collection - User profiles with base\_role and capabilities
    \item \texttt{capability\_requests} collection - Base role and capability approval workflows
    \item \texttt{api\_keys} collection - API keys valid across entire platform
    \item \texttt{audit\_logs} collection - Complete audit trail for all components
\end{itemize}

\section{Role \& Permission Model v3.0}

\subsection{Base Roles}

\subsubsection{Administrator}
\textbf{Purpose}: System-level authority and platform configuration

\textbf{Permissions}:
\begin{itemize}
    \item Full system access across all components
    \item User management (approve roles and capabilities)
    \item System configuration
    \item Day Zero knowledge seeding
    \item Override any review decision
    \item Access all audit logs
\end{itemize}

\subsubsection{Knowledge Curator}
\textbf{Purpose}: Content creator and domain expert

\textbf{Base Permissions}:
\begin{itemize}
    \item Upload documents (single and batch)
    \item Create facts and stylized facts
    \item Build knowledge graphs
    \item Create scenarios and models (in MA)
    \item Edit own contributions
    \item View published knowledge
\end{itemize}

\textbf{Optional Capabilities} (must be requested and approved):
\begin{itemize}
    \item \textbf{Agent Access}: Run AI agents, use custom tools, view agent logs
    \item \textbf{Analytics Access}: Advanced queries, bulk export, API key generation
    \item \textbf{Reviewer Status}: Review queue access, approve/reject knowledge, quality control
\end{itemize}

\subsubsection{Knowledge Explorator}
\textbf{Purpose}: Read-only user for browsing knowledge

\textbf{Permissions}:
\begin{itemize}
    \item Browse and search published knowledge
    \item View knowledge graphs
    \item View published models and scenarios (in MA)
    \item Create private annotations
    \item Export limited datasets (for personal use)
    \item Save search queries
\end{itemize}

\subsection{Capability Request Workflow}

\subsubsection{New User - Base Role Request}
\begin{enumerate}
    \item User signs in with Google
    \item Status: \texttt{pending\_approval}, base\_role: \texttt{null}
    \item User fills role request form (choose Curator or Explorator)
    \item Provides affiliation, research area, justification
    \item Administrator reviews and approves/rejects
    \item User receives email notification
    \item Access granted based on approved base role
\end{enumerate}

\subsubsection{Existing Curator - Capability Request}
\begin{enumerate}
    \item Curator logs in with base access
    \item Profile page shows "Request additional capabilities"
    \item User selects desired capabilities:
    \begin{itemize}
        \item Agent Access
        \item Analytics Access
        \item Reviewer Status
    \end{itemize}
    \item Provides justification for each capability
    \item Administrator reviews
    \item Capabilities added to user profile
    \item User gains new permissions immediately
\end{enumerate}

\subsection{Permission Resolution}

Permissions are computed based on:
\begin{equation}
\text{User Permissions} = \text{Base Role Permissions} \cup \text{Capability Permissions}
\end{equation}

\textbf{Examples}:
\begin{itemize}
    \item Curator (base only): Can create/edit knowledge
    \item Curator + Agent Access: Can create knowledge AND run agents
    \item Curator + Analytics Access + Reviewer Status: Can create, export, AND review
    \item Administrator: Has all permissions regardless of capabilities
\end{itemize}

\section{Current Phase: Phase 0 Development Plan}

\subsection{Phase Overview}

\textbf{Goal}: Establish complete authentication, authorization, and user management infrastructure for the entire AdvanDEB platform.

\textbf{Deliverable}: Fully authenticated platform with 3 base roles + 3 capabilities, unified SSO across KB and MA, Google OAuth integration, API keys, knowledge review workflow, and Day Zero seeding capability.

\subsection{Implementation Plan}

\subsubsection{Stage 1: Foundation}
\textbf{Focus}: Backend authentication system

\textbf{Tasks}:
\begin{enumerate}
    \item Create \texttt{advandeb-shared-utils} repository and package structure
    \item Implement JWT token generation and validation
    \item Implement Google OAuth 2.0 client
    \item Create User, CapabilityRequest, APIKey, AuditLog models (Pydantic)
    \item Implement permission checking functions (has\_base\_role, has\_capability)
    \item Set up MongoDB connection utilities
    \item Create audit logging functions
    \item Write unit tests for auth utilities
    \item Set up CI/CD for shared package
\end{enumerate}

\textbf{Deliverable}: Functional \texttt{advandeb-shared-utils} package ready for integration

\subsubsection{Stage 2: User Management Backend}
\textbf{Focus}: Knowledge Builder backend integration

\textbf{Tasks}:
\begin{enumerate}
    \item Add \texttt{advandeb-shared-utils} dependency to KB backend
    \item Create \texttt{/auth} router (login, callback, logout, refresh)
    \item Create \texttt{/users} router (profile, update)
    \item Create \texttt{/capability-requests} router (create, list, approve/reject)
    \item Create \texttt{/api-keys} router (generate, list, revoke)
    \item Implement AuthMiddleware for all existing routes
    \item Implement RateLimiter middleware
    \item Implement AuditLogger middleware
    \item Create UserService, RoleService, APIKeyService
    \item Set up MongoDB collections: users, capability\_requests, api\_keys, audit\_logs
    \item Configure Google OAuth credentials
    \item Implement email notification system
\end{enumerate}

\textbf{Deliverable}: Fully functional backend authentication and user management

\subsubsection{Stage 3: Frontend Integration}
\textbf{Focus}: Knowledge Builder frontend

\textbf{Tasks}:
\begin{enumerate}
    \item Create Login View with Google OAuth button
    \item Create Profile View (display user info, base role, capabilities)
    \item Create Role Request View (for new users)
    \item Create Capability Request View (for existing curators)
    \item Create Administrator Dashboard (user list, pending requests)
    \item Create API Key Management View (generate, view, revoke)
    \item Implement Auth Store (Pinia/Vuex) with token management
    \item Add Axios interceptors for JWT token injection
    \item Add automatic token refresh logic
    \item Update all existing views with permission-based rendering
    \item Add error handling for 401/403 responses
    \item Implement "Request Access" prompts for insufficient permissions
\end{enumerate}

\textbf{Deliverable}: Complete authenticated frontend with SSO

\subsubsection{Stage 4: Review Workflow}
\textbf{Focus}: Knowledge validation system

\textbf{Tasks}:
\begin{enumerate}
    \item Add \texttt{status} field to all knowledge entities (facts, stylized\_facts, graphs, documents)
    \item Implement status transitions: draft → pending\_review → published/rejected/changes\_requested
    \item Create \texttt{/reviews} router (queue, approve, reject, request-changes)
    \item Create ReviewService with business logic
    \item Create Review Queue View (for users with Reviewer Status capability)
    \item Add status badges to knowledge list views
    \item Implement reviewer assignment logic
    \item Add review history tracking
    \item Create reviewer dashboard with statistics
    \item Add email notifications for review status changes
    \item Prevent self-review (users cannot review own contributions)
\end{enumerate}

\textbf{Deliverable}: Functional peer review system for knowledge quality control

\subsubsection{Stage 5: Day Zero \& Migration}
\textbf{Focus}: Initial knowledge seeding and data migration

\textbf{Tasks}:
\begin{enumerate}
    \item Create Day Zero batch ingestion workflow
    \item Add \texttt{is\_day\_zero} flag to knowledge entities
    \item Implement admin-only Day Zero creation endpoints
    \item Add "Foundational Knowledge" badges in UI
    \item Migrate existing 1,300 PDFs from /papers directory
    \item Create migration script for legacy data
    \item Add attribution metadata to migrated content
    \item Auto-approve Day Zero content (skip review)
    \item Create Day Zero management dashboard
    \item Add bulk tagging for Day Zero content
    \item Test and validate all migrated data
\end{enumerate}

\textbf{Deliverable}: Platform seeded with foundational knowledge, legacy data migrated

\subsubsection{Stage 6: MA Integration \& Polish}
\textbf{Focus}: Modeling Assistant authentication and final testing

\textbf{Tasks}:
\begin{enumerate}
    \item Add \texttt{advandeb-shared-utils} dependency to MA backend
    \item Implement authentication middleware in MA using shared library
    \item Add JWT token validation to all MA routes
    \item Implement permission checks for MA-specific operations (create scenarios, run simulations)
    \item Update MA frontend to use shared Auth Store
    \item Test cross-component authentication (KB → MA with same token)
    \item Add component field to audit logs ("knowledge\_builder" vs "modeling\_assistant")
    \item Create comprehensive integration tests
    \item Perform security audit (token expiration, permission boundaries, rate limiting)
    \item Load testing (authenticate 100+ concurrent users)
    \item Write user documentation (authentication guide, capability request guide)
    \item Write administrator documentation (user management, approval workflows)
    \item Write developer documentation (adding new permissions, extending capabilities)
    \item Final bug fixes and polish
\end{enumerate}

\textbf{Deliverable}: Unified platform with complete authentication across KB and MA

\section{Technical Implementation Details}

\subsection{Database Schema}

\subsubsection{users Collection}
\begin{verbatim}
{
  "_id": ObjectId,
  "google_id": string,          // Unique
  "email": string,
  "name": string,
  "picture_url": string,
  "base_role": string,          // "administrator", "knowledge_curator", 
                                //  "knowledge_explorator"
  "capabilities": [string],     // ["agent_access", "analytics_access", 
                                //  "reviewer_status"]
  "status": string,             // "active", "suspended", "pending_approval"
  "created_at": datetime,
  "updated_at": datetime,
  "last_login": datetime,
  "login_count": int,
  "metadata": {
    "affiliation": string,
    "research_area": string,
    "orcid": string
  }
}
\end{verbatim}

\subsubsection{capability\_requests Collection}
\begin{verbatim}
{
  "_id": ObjectId,
  "user_id": ObjectId,
  "request_type": string,       // "base_role" or "capability"
  
  // For base role requests
  "requested_base_role": string,
  "current_base_role": string,
  
  // For capability requests
  "requested_capabilities": [string],
  "current_capabilities": [string],
  
  "justification": string,
  "form_data": dict,
  "status": string,             // "pending", "approved", "rejected"
  "created_at": datetime,
  "reviewed_by": ObjectId,
  "reviewed_at": datetime,
  "review_notes": string
}
\end{verbatim}

\subsubsection{api\_keys Collection}
\begin{verbatim}
{
  "_id": ObjectId,
  "user_id": ObjectId,
  "key_hash": string,           // SHA-256 of plain key
  "key_prefix": string,         // "advk_abc12345"
  "name": string,
  "scopes": [string],           // Auto-assigned based on user's 
                                // base_role + capabilities
  "status": string,             // "active", "revoked", "expired"
  "created_at": datetime,
  "expires_at": datetime,
  "last_used_at": datetime,
  "rate_limit": {
    "requests_per_minute": int,
    "requests_per_day": int
  }
}
\end{verbatim}

\subsubsection{audit\_logs Collection}
\begin{verbatim}
{
  "_id": ObjectId,
  "user_id": ObjectId,
  "action": string,             // "create_fact", "approve_knowledge", etc.
  "resource_type": string,      // "fact", "document", "scenario", etc.
  "resource_id": ObjectId,
  "component": string,          // "knowledge_builder" or "modeling_assistant"
  "details": dict,
  "ip_address": string,
  "user_agent": string,
  "auth_method": string,        // "jwt", "api_key"
  "timestamp": datetime
}
\end{verbatim}

\subsection{API Endpoints}

\subsubsection{Authentication Routes}
\begin{itemize}
    \item \texttt{GET /auth/login} - Redirect to Google OAuth
    \item \texttt{GET /auth/callback} - OAuth callback handler
    \item \texttt{POST /auth/logout} - Invalidate tokens
    \item \texttt{POST /auth/refresh} - Refresh access token
\end{itemize}

\subsubsection{User Management Routes}
\begin{itemize}
    \item \texttt{GET /users/me} - Get current user profile
    \item \texttt{PATCH /users/me} - Update profile
    \item \texttt{GET /users} - List users (admin only)
    \item \texttt{GET /users/:id} - Get user details (admin only)
\end{itemize}

\subsubsection{Capability Request Routes}
\begin{itemize}
    \item \texttt{POST /capability-requests} - Create new request
    \item \texttt{GET /capability-requests} - List own requests
    \item \texttt{GET /capability-requests/pending} - List pending (admin only)
    \item \texttt{POST /capability-requests/:id/approve} - Approve (admin only)
    \item \texttt{POST /capability-requests/:id/reject} - Reject (admin only)
\end{itemize}

\subsubsection{API Key Routes}
\begin{itemize}
    \item \texttt{POST /api-keys} - Generate new key
    \item \texttt{GET /api-keys} - List own keys
    \item \texttt{DELETE /api-keys/:id} - Revoke key
\end{itemize}

\subsubsection{Review Routes}
\begin{itemize}
    \item \texttt{GET /reviews/queue} - Pending review items (requires Reviewer Status)
    \item \texttt{POST /reviews/:id/approve} - Approve knowledge
    \item \texttt{POST /reviews/:id/reject} - Reject knowledge
    \item \texttt{POST /reviews/:id/request-changes} - Request changes
\end{itemize}

\subsection{Security Considerations}

\subsubsection{Token Security}
\begin{itemize}
    \item JWT secret key: 256-bit random value stored in environment
    \item Access tokens: Short-lived (1 hour) to limit exposure window
    \item Refresh tokens: Longer-lived (30 days) but stored securely
    \item Token rotation on refresh to prevent replay attacks
    \item JTI (JWT ID) for token revocation capability
\end{itemize}

\subsubsection{API Key Security}
\begin{itemize}
    \item SHA-256 hashing before storage (plain key never stored)
    \item Prefix-based identification (\texttt{advk\_...}) for quick lookup
    \item Rate limiting: 200-500 requests/minute depending on capabilities
    \item Automatic expiration (30-90 days)
    \item IP whitelist support (optional)
\end{itemize}

\subsubsection{Rate Limiting}
\begin{itemize}
    \item Per-user rate limits based on capabilities
    \item Redis-backed token bucket algorithm
    \item Limits: Explorator (100 req/min), Curator (200 req/min), Curator+Analytics (500 req/min)
    \item Daily limits for bulk operations
\end{itemize}

\subsubsection{Audit Logging}
\begin{itemize}
    \item All write operations logged
    \item All authentication events logged
    \item All permission changes logged
    \item Immutable logs (append-only)
    \item Retention: 2 years minimum
\end{itemize}

\section{Testing Strategy}

\subsection{Unit Tests}
\begin{itemize}
    \item All functions in \texttt{advandeb-shared-utils}
    \item JWT generation and validation
    \item Permission checking logic
    \item API key hashing and validation
    \item Target coverage: 90\%+
\end{itemize}

\subsection{Integration Tests}
\begin{itemize}
    \item Complete authentication flows (OAuth, JWT, API keys)
    \item Cross-component authentication (KB → MA)
    \item Permission enforcement on all protected routes
    \item Capability request workflow
    \item Review workflow
\end{itemize}

\subsection{End-to-End Tests}
\begin{itemize}
    \item User registration through approval
    \item Curator requests capabilities
    \item Knowledge creation and review
    \item Cross-component access (create in KB, view in MA)
\end{itemize}

\subsection{Security Testing}
\begin{itemize}
    \item Attempt access without authentication
    \item Attempt privilege escalation
    \item Token expiration validation
    \item Rate limiting enforcement
    \item SQL injection attempts (though using MongoDB)
    \item XSS vulnerability testing
\end{itemize}

\section{Deployment Plan}

\subsection{Development Environment}
\begin{itemize}
    \item Local MongoDB instance
    \item Local Ollama for LLM functionality
    \item Google OAuth test credentials
    \item Mock email service
\end{itemize}

\subsection{Staging Environment}
\begin{itemize}
    \item Hosted MongoDB (Atlas)
    \item Deployed backend (Docker containers)
    \item Deployed frontend (Nginx)
    \item Real Google OAuth credentials (test domain)
    \item Real email service (SendGrid/Mailgun)
\end{itemize}

\subsection{Production Environment}
\begin{itemize}
    \item Production MongoDB cluster (replica set)
    \item Redis cluster for rate limiting
    \item Load-balanced backend servers
    \item CDN for frontend assets
    \item SSL/TLS certificates
    \item Backup and disaster recovery
    \item Monitoring and alerting
\end{itemize}

\section{Risk Assessment}

\subsection{Technical Risks}

\subsubsection{Risk: Complex Permission Logic}
\textbf{Probability}: Medium\\
\textbf{Impact}: High\\
\textbf{Mitigation}:
\begin{itemize}
    \item Comprehensive unit tests for permission functions
    \item Permission matrix documentation
    \item Code review focus on authorization code
\end{itemize}

\subsubsection{Risk: OAuth Integration Issues}
\textbf{Probability}: Low\\
\textbf{Impact}: High\\
\textbf{Mitigation}:
\begin{itemize}
    \item Early integration testing
    \item Fallback to email/password authentication if needed
    \item Well-documented OAuth configuration
\end{itemize}

\subsubsection{Risk: Performance Bottlenecks}
\textbf{Probability}: Medium\\
\textbf{Impact}: Medium\\
\textbf{Mitigation}:
\begin{itemize}
    \item Load testing during final stages
    \item Redis caching for frequently accessed data
    \item Database indexing on user lookups
\end{itemize}

\subsection{Project Risks}

\subsubsection{Risk: Scope Creep}
\textbf{Probability}: Medium\\
\textbf{Impact}: High\\
\textbf{Mitigation}:
\begin{itemize}
    \item Strict adherence to Phase 0 scope
    \item Feature requests deferred to Phase 1
    \item Weekly progress reviews
\end{itemize}

\subsubsection{Risk: Project Delays}
\textbf{Probability}: Medium\\
\textbf{Impact}: Medium\\
\textbf{Mitigation}:
\begin{itemize}
    \item Regular milestone tracking
    \item Early identification of blockers
    \item Flexible resource allocation
\end{itemize}

\section{Success Criteria}

Phase 0 will be considered complete when:

\begin{enumerate}
    \item All authentication methods functional (Google OAuth, JWT, API keys)
    \item All 3 base roles + 3 capabilities implemented and tested
    \item Capability request workflow functional for both base roles and capabilities
    \item Knowledge review workflow operational
    \item Existing 1,300 PDFs migrated with Day Zero attribution
    \item Cross-component authentication verified (KB $\leftrightarrow$ MA)
    \item Complete audit logging operational
    \item Security audit passed
    \item User documentation complete
    \item Administrator documentation complete
    \item All unit, integration, and E2E tests passing
    \item System handles 100+ concurrent users
\end{enumerate}

\section{Next Phases}

Following successful completion of Phase 0:

\subsection{Phase 1: Knowledge Builder Stabilization}
\begin{itemize}
    \item Harden knowledge CRUD operations with permissions
    \item Stabilize agent framework with user attribution
    \item Add comprehensive test coverage
    \item Implement CI/CD pipeline
    \item Performance optimization
\end{itemize}

\subsection{Phase 2: Modeling Assistant Prototype}
\begin{itemize}
    \item Finalize MA knowledge integration contracts
    \item Implement scenario builder with permissions
    \item Create model assembly interface
    \item Validate end-to-end flow: knowledge → modeling
\end{itemize}

\subsection{Phase 3: Enhanced Integration \& UX}
\begin{itemize}
    \item Advanced search tailored for modeling use cases
    \item Visual knowledge exploration in MA
    \item Collaboration features (shared scenarios)
    \item Multi-user contribution tracking
\end{itemize}

\subsection{Phase 4: Extensions \& Plugins}
\begin{itemize}
    \item Plugin mechanism for custom tools
    \item Support for additional modeling paradigms
    \item Advanced collaboration (workspaces, teams)
    \item Trust and reputation system
\end{itemize}

\section{Conclusion}

Phase 0 represents the critical foundation for the AdvanDEB Platform, establishing security, access control, and collaboration infrastructure. The simplified v3.0 capability-based architecture provides flexibility while reducing complexity compared to previous designs.

Upon completion, the platform will have a robust, production-ready authentication system that scales to support future phases.

The capability-based model allows users to grow their access as their needs evolve, supporting the platform's goal of fostering collaboration while maintaining quality control through the review workflow.

\vspace{1cm}

\noindent\textbf{Document Version}: 3.0\\
\textbf{Date}: December 12, 2025\\
\textbf{Status}: Ready for Implementation\\
\textbf{Contact}: AdvanDEB Development Team

\end{document}
